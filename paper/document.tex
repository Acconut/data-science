\documentclass[sigconf]{acmart}

\usepackage{booktabs} % For formal tables
\usepackage{hyperref}


% Copyright
%\setcopyright{none}
%\setcopyright{acmcopyright}
%\setcopyright{acmlicensed}
\setcopyright{rightsretained}
%\setcopyright{usgov}
%\setcopyright{usgovmixed}
%\setcopyright{cagov}
%\setcopyright{cagovmixed}


% DOI
% \acmDOI{10.475/123_4}

% ISBN
% \acmISBN{123-4567-24-567/08/06}

%Conference
\acmConference[UPDS-SoSe18]{Uni Passau - Data Science Seminar - SoSe18}{July 2018}{Passau, Germany}
\copyrightyear{2018}

\begin{document}
\title{Hyperparameter Study for Classical Feedforward Neural Networks on Text Data}

\author{Marius Kleidl}
\affiliation{%
  \institution{University of Passau}
  \city{Passau}
}
\email{kleidl01@gw.uni-passau.de}


% The default list of authors is too long for headers.
\renewcommand{\shortauthors}{B. Trovato et al.}


\begin{abstract}
This paper provides a sample of a \LaTeX\ document which conforms,
somewhat loosely, to the formatting guidelines for
ACM SIG Proceedings.
\end{abstract}

%
% The code below should be generated by the tool at
% http://dl.acm.org/ccs.cfm
% Please copy and paste the code instead of the example below.
%
 \begin{CCSXML}
	<ccs2012>
	<concept>
	<concept_id>10010147.10010257.10010258.10010259.10010263</concept_id>
	<concept_desc>Computing methodologies~Supervised learning by classification</concept_desc>
	<concept_significance>500</concept_significance>
	</concept>
	<concept>
	<concept_id>10010147.10010257.10010293.10010294</concept_id>
	<concept_desc>Computing methodologies~Neural networks</concept_desc>
	<concept_significance>500</concept_significance>
	</concept>
	</ccs2012>
\end{CCSXML}

\ccsdesc[500]{Computing methodologies~Supervised learning by classification}
\ccsdesc[500]{Computing methodologies~Neural networks}


\keywords{ACM proceedings, \LaTeX, text tagging}


\maketitle

\section{Introduction}

\section{Data Set}

The experiments are performed in the well-known 20Newsgroups\footnote{\href{http://qwone.com/~jason/20Newsgroups/}{http://qwone.com/$\sim$jason/20Newsgroups/}} data set. It contains 18846 postings which are fairly evenly distributed across 20 different newsgroups, as can be seen in Table~\ref{tab:groups}. Furthermore, the set has been split into a training set of 11314 postings and a testing set of 7532.

TODO: message format 

\begin{table}[]
	\centering
	\caption{Postings per newsgroup}
	\label{tab:groups}
	\begin{tabular}{ccl}
		\hline                      %inserts double horizontal lines
		Newsgroup & Number of postings \\ %[0.5ex]
		\hline
rec.motorcycles & 996 & \\
comp.sys.mac.hardware & 963 & \\
talk.politics.misc & 775 & \\
soc.religion.christian & 997 & \\
comp.graphics & 973 & \\
sci.med & 990 & \\
talk.religion.misc & 628 & \\
comp.windows.x & 988 & \\
comp.sys.ibm.pc.hardware & 982 & \\
talk.politics.guns & 910 & \\
alt.atheism & 799 & \\
comp.os.ms-windows.misc & 985 & \\
sci.crypt & 991 & \\
sci.space & 987 & \\
misc.forsale & 975 & \\
rec.sport.hockey & 999 & \\
rec.sport.baseball & 994 & \\
sci.electronics & 984 & \\
rec.autos & 990 & \\
talk.politics.mideast & 940 &
	\end{tabular}
\end{table}

\section{Experiment}

\section{Results}

\section{Conclusion}

\bibliographystyle{ACM-Reference-Format}
\bibliography{sample-bibliography}

\end{document}
